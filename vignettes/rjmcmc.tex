\documentclass[ignorenonframetext,]{beamer}
\setbeamertemplate{caption}[numbered]
\setbeamertemplate{caption label separator}{: }
\setbeamercolor{caption name}{fg=normal text.fg}
\beamertemplatenavigationsymbolsempty
\usepackage{lmodern}
\usepackage{amssymb,amsmath}
\usepackage{ifxetex,ifluatex}
\usepackage{fixltx2e} % provides \textsubscript
\ifnum 0\ifxetex 1\fi\ifluatex 1\fi=0 % if pdftex
\usepackage[T1]{fontenc}
\usepackage[utf8]{inputenc}
\else % if luatex or xelatex
\ifxetex
\usepackage{mathspec}
\else
\usepackage{fontspec}
\fi
\defaultfontfeatures{Ligatures=TeX,Scale=MatchLowercase}
\fi
% use upquote if available, for straight quotes in verbatim environments
\IfFileExists{upquote.sty}{\usepackage{upquote}}{}
% use microtype if available
\IfFileExists{microtype.sty}{%
\usepackage{microtype}
\UseMicrotypeSet[protrusion]{basicmath} % disable protrusion for tt fonts
}{}
\newif\ifbibliography
\usepackage{color}
\usepackage{fancyvrb}
\newcommand{\VerbBar}{|}
\newcommand{\VERB}{\Verb[commandchars=\\\{\}]}
\DefineVerbatimEnvironment{Highlighting}{Verbatim}{commandchars=\\\{\}}
% Add ',fontsize=\small' for more characters per line
\usepackage{framed}
\definecolor{shadecolor}{RGB}{248,248,248}
\newenvironment{Shaded}{\begin{snugshade}}{\end{snugshade}}
\newcommand{\KeywordTok}[1]{\textcolor[rgb]{0.13,0.29,0.53}{\textbf{{#1}}}}
\newcommand{\DataTypeTok}[1]{\textcolor[rgb]{0.13,0.29,0.53}{{#1}}}
\newcommand{\DecValTok}[1]{\textcolor[rgb]{0.00,0.00,0.81}{{#1}}}
\newcommand{\BaseNTok}[1]{\textcolor[rgb]{0.00,0.00,0.81}{{#1}}}
\newcommand{\FloatTok}[1]{\textcolor[rgb]{0.00,0.00,0.81}{{#1}}}
\newcommand{\ConstantTok}[1]{\textcolor[rgb]{0.00,0.00,0.00}{{#1}}}
\newcommand{\CharTok}[1]{\textcolor[rgb]{0.31,0.60,0.02}{{#1}}}
\newcommand{\SpecialCharTok}[1]{\textcolor[rgb]{0.00,0.00,0.00}{{#1}}}
\newcommand{\StringTok}[1]{\textcolor[rgb]{0.31,0.60,0.02}{{#1}}}
\newcommand{\VerbatimStringTok}[1]{\textcolor[rgb]{0.31,0.60,0.02}{{#1}}}
\newcommand{\SpecialStringTok}[1]{\textcolor[rgb]{0.31,0.60,0.02}{{#1}}}
\newcommand{\ImportTok}[1]{{#1}}
\newcommand{\CommentTok}[1]{\textcolor[rgb]{0.56,0.35,0.01}{\textit{{#1}}}}
\newcommand{\DocumentationTok}[1]{\textcolor[rgb]{0.56,0.35,0.01}{\textbf{\textit{{#1}}}}}
\newcommand{\AnnotationTok}[1]{\textcolor[rgb]{0.56,0.35,0.01}{\textbf{\textit{{#1}}}}}
\newcommand{\CommentVarTok}[1]{\textcolor[rgb]{0.56,0.35,0.01}{\textbf{\textit{{#1}}}}}
\newcommand{\OtherTok}[1]{\textcolor[rgb]{0.56,0.35,0.01}{{#1}}}
\newcommand{\FunctionTok}[1]{\textcolor[rgb]{0.00,0.00,0.00}{{#1}}}
\newcommand{\VariableTok}[1]{\textcolor[rgb]{0.00,0.00,0.00}{{#1}}}
\newcommand{\ControlFlowTok}[1]{\textcolor[rgb]{0.13,0.29,0.53}{\textbf{{#1}}}}
\newcommand{\OperatorTok}[1]{\textcolor[rgb]{0.81,0.36,0.00}{\textbf{{#1}}}}
\newcommand{\BuiltInTok}[1]{{#1}}
\newcommand{\ExtensionTok}[1]{{#1}}
\newcommand{\PreprocessorTok}[1]{\textcolor[rgb]{0.56,0.35,0.01}{\textit{{#1}}}}
\newcommand{\AttributeTok}[1]{\textcolor[rgb]{0.77,0.63,0.00}{{#1}}}
\newcommand{\RegionMarkerTok}[1]{{#1}}
\newcommand{\InformationTok}[1]{\textcolor[rgb]{0.56,0.35,0.01}{\textbf{\textit{{#1}}}}}
\newcommand{\WarningTok}[1]{\textcolor[rgb]{0.56,0.35,0.01}{\textbf{\textit{{#1}}}}}
\newcommand{\AlertTok}[1]{\textcolor[rgb]{0.94,0.16,0.16}{{#1}}}
\newcommand{\ErrorTok}[1]{\textcolor[rgb]{0.64,0.00,0.00}{\textbf{{#1}}}}
\newcommand{\NormalTok}[1]{{#1}}
\usepackage{graphicx,grffile}
\makeatletter
\def\maxwidth{\ifdim\Gin@nat@width>\linewidth\linewidth\else\Gin@nat@width\fi}
\def\maxheight{\ifdim\Gin@nat@height>\textheight0.8\textheight\else\Gin@nat@height\fi}
\makeatother
% Scale images if necessary, so that they will not overflow the page
% margins by default, and it is still possible to overwrite the defaults
% using explicit options in \includegraphics[width, height, ...]{}
\setkeys{Gin}{width=\maxwidth,height=\maxheight,keepaspectratio}

% Prevent slide breaks in the middle of a paragraph:
\widowpenalties 1 10000
\raggedbottom

\AtBeginPart{
\let\insertpartnumber\relax
\let\partname\relax
\frame{\partpage}
}
\AtBeginSection{
\ifbibliography
\else
\let\insertsectionnumber\relax
\let\sectionname\relax
\frame{\sectionpage}
\fi
}
\AtBeginSubsection{
\let\insertsubsectionnumber\relax
\let\subsectionname\relax
\frame{\subsectionpage}
}

\setlength{\parindent}{0pt}
\setlength{\parskip}{6pt plus 2pt minus 1pt}
\setlength{\emergencystretch}{3em}  % prevent overfull lines
\providecommand{\tightlist}{%
\setlength{\itemsep}{0pt}\setlength{\parskip}{0pt}}
\setcounter{secnumdepth}{0}

\title{Bayesianische Variablenselektion}
\author{Volker Schmid}
\date{\begin{enumerate}
\def\labelenumi{\arabic{enumi}.}
\setcounter{enumi}{9}
\tightlist
\item
  Juli 2017
\end{enumerate}}

\begin{document}
\frame{\titlepage}

\begin{frame}
\tableofcontents[hideallsubsections]
\end{frame}

\section{Reversible Jump MCMC
(RJMCMC)}\label{reversible-jump-mcmc-rjmcmc}

\begin{frame}{Modellwahl/Variablenwahl mit unterschiedlichen
Dimensionen}

Vergleichen wir zwei unterschiedliche Modelle

\begin{enumerate}
\def\labelenumi{\arabic{enumi}.}
\tightlist
\item
  \(y = \alpha + \beta x + \epsilon\)
\item
  \(y = \alpha + \epsilon\)
\end{enumerate}

so lässt sich die Modellwahl auch als Variablenselektion mit
Indikatorvariablen interpretieren. Im zweiten Modell ist die
Indikatorvariable für \(\beta\) gleich 0.

Im MCMC-Algorithmus sind also identisch:

\begin{itemize}
\tightlist
\item
  \(I=1 \to I=0\)
\item
  \(\beta \to 0\)
\item
  Modell 1 \(\to\) Modell 2
\item
  \(\theta=(\alpha,\beta) \to \theta^*=(\alpha)\)
\end{itemize}

\end{frame}

\begin{frame}[allowframebreaks]{Reversible Jump MCMC}

\begin{itemize}
\tightlist
\item
  RJMCMC nach Green (1995)
\item
  Allgemein sind bei Reversible Jump verschiedene Parameterräume erlaubt
\item
  Zwischen den Parameterräumen müssen Abbildungen (reversible jumps)
  möglich sein
\item
  Im obigen Beispiel:
\end{itemize}

\textbf{death step} \[ (\alpha,\beta) = (\alpha)\]

\textbf{birth step}
\[(\alpha) \to (\alpha,\beta) \text{ mit }\beta\sim \text{prior}\]

\begin{itemize}
\tightlist
\item
  Der Modellwechsel geht in die Akzeptanzwahrscheinlichkeit ein (auch
  \emph{Metropolis-Hastings-Green}-Wahrscheinlichkeit).
\item
  Mit \(\theta\) alter Zustand und \(\theta^*\) Vorschlag
\end{itemize}

\[
\alpha = \frac{f(y|\theta^*)}{f(y|\theta)}\frac{p(\theta^*)}{p(\theta)}
\frac{q(\theta|\theta^*)}{q(\theta^*|\theta)}\left|J\right|
\]

wobei \(J\) die Jacobi-Matrix für den deterministischen Übergang von
\(\theta \to \theta^*\) ist

\end{frame}

\begin{frame}[fragile,allowframebreaks]{RJMCMC Beispiel}

\begin{Shaded}
\begin{Highlighting}[]
\NormalTok{g.predict<-function(Mod, X)}
\NormalTok{\{}
\NormalTok{j     =}\StringTok{ }\NormalTok{Mod[}\DecValTok{2}\NormalTok{]}
\NormalTok{beta0 =}\StringTok{ }\NormalTok{Mod[}\DecValTok{3}\NormalTok{]}
\NormalTok{beta1 =}\StringTok{ }\NormalTok{Mod[}\DecValTok{4}\NormalTok{]}
\NormalTok{beta2 =}\StringTok{ }\NormalTok{Mod[}\DecValTok{5}\NormalTok{]}

\NormalTok{P =}\StringTok{ }\DecValTok{0}\NormalTok{*X +}\StringTok{ }\NormalTok{beta0}
\NormalTok{if (j >=}\StringTok{ }\DecValTok{2}\NormalTok{) \{P =}\StringTok{ }\NormalTok{P +}\StringTok{ }\NormalTok{X*beta1\}}
\NormalTok{if (j >=}\StringTok{ }\DecValTok{3}\NormalTok{) \{P =}\StringTok{ }\NormalTok{P +}\StringTok{ }\NormalTok{(X^}\DecValTok{2}\NormalTok{)*beta2\}}

\KeywordTok{return}\NormalTok{(P)}
\NormalTok{\}}
\end{Highlighting}
\end{Shaded}

\begin{Shaded}
\begin{Highlighting}[]
\NormalTok{g.perterb<-function(}\DataTypeTok{M=}\KeywordTok{c}\NormalTok{(-}\OtherTok{Inf}\NormalTok{, }\DecValTok{3}\NormalTok{, }\DecValTok{0}\NormalTok{, }\DecValTok{0}\NormalTok{, }\DecValTok{0}\NormalTok{), }\DataTypeTok{Qsd=}\KeywordTok{c}\NormalTok{(}\DecValTok{0}\NormalTok{, }\DecValTok{0}\NormalTok{, }\FloatTok{0.1}\NormalTok{, }\FloatTok{0.01}\NormalTok{, }\FloatTok{0.001}\NormalTok{), }\DataTypeTok{LB =} \KeywordTok{c}\NormalTok{(}\DecValTok{0}\NormalTok{, }\DecValTok{1}\NormalTok{, -}\DecValTok{10}\NormalTok{, -}\DecValTok{2}\NormalTok{, -}\DecValTok{1} \NormalTok{), }\DataTypeTok{UB =} \KeywordTok{c}\NormalTok{(}\DecValTok{0}\NormalTok{, }\DecValTok{1}\NormalTok{, }\DecValTok{10}\NormalTok{, }\DecValTok{2}\NormalTok{, }\DecValTok{1} \NormalTok{)  , }\DataTypeTok{data=}\KeywordTok{data.frame}\NormalTok{(}\DecValTok{1}\NormalTok{:}\DecValTok{100}\NormalTok{, }\KeywordTok{rnorm}\NormalTok{(}\DecValTok{100}\NormalTok{)))}
\NormalTok{\{}
\CommentTok{# unpacking hte parameters}
\NormalTok{LL    =}\StringTok{ }\NormalTok{M[}\DecValTok{1}\NormalTok{]}
\NormalTok{j     =}\StringTok{ }\NormalTok{M[}\DecValTok{2}\NormalTok{]}
\CommentTok{#beta0 = M[3]}
\CommentTok{#beta1 = M[4]}
\CommentTok{#beta2 = M[5]}
\NormalTok{x     =}\StringTok{ }\NormalTok{data[,}\DecValTok{1}\NormalTok{]}
\NormalTok{y     =}\StringTok{ }\NormalTok{data[,}\DecValTok{2}\NormalTok{]}

\NormalTok{ORDER =}\StringTok{ }\KeywordTok{sample}\NormalTok{(}\DecValTok{3}\NormalTok{:(}\DecValTok{3}\NormalTok{+j}\DecValTok{-1}\NormalTok{), j)}

\NormalTok{for (i in ORDER)}
\NormalTok{\{}
\NormalTok{M.prime =}\StringTok{ }\NormalTok{M                                           }\CommentTok{# make the proposal model}
\NormalTok{M.prime[i] =}\StringTok{ }\NormalTok{M.prime[i] +}\StringTok{ }\KeywordTok{rnorm}\NormalTok{(}\DecValTok{1}\NormalTok{, }\DataTypeTok{mean =} \DecValTok{0}\NormalTok{, }\DataTypeTok{sd=} \NormalTok{Qsd[i]) }\CommentTok{# add random noise to old model}
\NormalTok{P =}\StringTok{ }\KeywordTok{g.predict}\NormalTok{(M.prime, x)                             }\CommentTok{# get predicted values}

\NormalTok{LL.prime =}\StringTok{ }\KeywordTok{sum}\NormalTok{(}\KeywordTok{dnorm}\NormalTok{(y-P, }\DataTypeTok{mean=} \DecValTok{0}\NormalTok{, }\DataTypeTok{sd=}\DecValTok{1}\NormalTok{, }\DataTypeTok{log=}\NormalTok{T))      }\CommentTok{# compute loglikihood}
\NormalTok{M.prime[}\DecValTok{1}\NormalTok{] =}\StringTok{ }\NormalTok{LL.prime                                 }\CommentTok{# save LL}

\NormalTok{r =}\StringTok{ }\KeywordTok{runif}\NormalTok{(}\DecValTok{1}\NormalTok{)                                          }\CommentTok{# random uniform}
\NormalTok{MH =}\StringTok{ }\KeywordTok{exp}\NormalTok{(LL.prime -}\StringTok{ }\NormalTok{LL)                               }\CommentTok{# Metropolis-hasting acceptance probability value}
\NormalTok{if ((r <=}\StringTok{ }\NormalTok{MH) &}\StringTok{ }\NormalTok{(M.prime[i] >=}\StringTok{ }\NormalTok{LB[i]) &}\StringTok{ }\NormalTok{(M.prime[i] <=}\StringTok{ }\NormalTok{UB[i])  ) \{M =}\StringTok{ }\NormalTok{M.prime\}                            }\CommentTok{# if accepted and in bounds}

\NormalTok{\}}
\KeywordTok{return}\NormalTok{(M)}
\NormalTok{\}}
\end{Highlighting}
\end{Shaded}

\begin{Shaded}
\begin{Highlighting}[]
\NormalTok{g.birth<-function(}\DataTypeTok{M=}\KeywordTok{c}\NormalTok{(-}\OtherTok{Inf}\NormalTok{, }\DecValTok{1}\NormalTok{, }\DecValTok{0}\NormalTok{, }\DecValTok{0}\NormalTok{, }\DecValTok{0}\NormalTok{), }\DataTypeTok{Qsd=}\KeywordTok{c}\NormalTok{(}\DecValTok{0}\NormalTok{, }\DecValTok{0}\NormalTok{, }\FloatTok{0.1}\NormalTok{, }\FloatTok{0.01}\NormalTok{, }\FloatTok{0.001}\NormalTok{), }\DataTypeTok{LB =} \KeywordTok{c}\NormalTok{(}\DecValTok{0}\NormalTok{, }\DecValTok{1}\NormalTok{, -}\DecValTok{10}\NormalTok{, -}\DecValTok{2}\NormalTok{, -}\DecValTok{1} \NormalTok{), }\DataTypeTok{UB =} \KeywordTok{c}\NormalTok{(}\DecValTok{0}\NormalTok{, }\DecValTok{1}\NormalTok{, }\DecValTok{10}\NormalTok{, }\DecValTok{2}\NormalTok{, }\DecValTok{1} \NormalTok{)  , }\DataTypeTok{data=}\KeywordTok{data.frame}\NormalTok{(}\DecValTok{1}\NormalTok{:}\DecValTok{100}\NormalTok{, }\KeywordTok{rnorm}\NormalTok{(}\DecValTok{100}\NormalTok{)))}
\NormalTok{\{}
\CommentTok{# unpacking hte parameters}
\NormalTok{LL    =}\StringTok{ }\NormalTok{M[}\DecValTok{1}\NormalTok{]}
\NormalTok{j     =}\StringTok{ }\NormalTok{M[}\DecValTok{2}\NormalTok{]}
\NormalTok{x     =}\StringTok{ }\NormalTok{data[,}\DecValTok{1}\NormalTok{]}
\NormalTok{y     =}\StringTok{ }\NormalTok{data[,}\DecValTok{2}\NormalTok{]}

\NormalTok{if (j ==}\StringTok{ }\DecValTok{1}\NormalTok{)}
\NormalTok{\{}
\NormalTok{M.prime =}\StringTok{ }\NormalTok{M                                           }\CommentTok{# make the proposal model}
\NormalTok{M.prime[}\DecValTok{2}\NormalTok{] =}\StringTok{ }\DecValTok{2}
\NormalTok{M.prime[}\DecValTok{4}\NormalTok{] =}\StringTok{ }\KeywordTok{runif}\NormalTok{(}\DecValTok{1}\NormalTok{, }\DataTypeTok{min =} \NormalTok{LB[}\DecValTok{4}\NormalTok{], }\DataTypeTok{max =} \NormalTok{UB[}\DecValTok{4}\NormalTok{])       }\CommentTok{# propose from prior}
\NormalTok{P =}\StringTok{ }\KeywordTok{g.predict}\NormalTok{(M.prime, x)                             }\CommentTok{# get predicted values}

\NormalTok{LL.prime =}\StringTok{ }\KeywordTok{sum}\NormalTok{(}\KeywordTok{dnorm}\NormalTok{(y-P, }\DataTypeTok{mean=} \DecValTok{0}\NormalTok{, }\DataTypeTok{sd=}\DecValTok{1}\NormalTok{, }\DataTypeTok{log=}\NormalTok{T))      }\CommentTok{# compute loglikihood}
\NormalTok{M.prime[}\DecValTok{1}\NormalTok{] =}\StringTok{ }\NormalTok{LL.prime                                 }\CommentTok{# save LL}

\NormalTok{r =}\StringTok{ }\KeywordTok{runif}\NormalTok{(}\DecValTok{1}\NormalTok{)                                          }\CommentTok{# random uniform}
\NormalTok{MH =}\StringTok{ }\KeywordTok{exp}\NormalTok{(LL.prime -}\StringTok{ }\NormalTok{LL)                               }\CommentTok{# Metropolis-hasting acceptance probability value}
\NormalTok{if (r <=}\StringTok{ }\NormalTok{MH) \{M =}\StringTok{ }\NormalTok{M.prime\}                            }\CommentTok{# if accepted}

\NormalTok{\}}

\NormalTok{if (j ==}\StringTok{ }\DecValTok{2}\NormalTok{)}
\NormalTok{\{}
\NormalTok{M.prime =}\StringTok{ }\NormalTok{M                                           }\CommentTok{# make the proposal model}
\NormalTok{M.prime[}\DecValTok{2}\NormalTok{] =}\StringTok{ }\DecValTok{3}
\NormalTok{M.prime[}\DecValTok{5}\NormalTok{] =}\StringTok{ }\KeywordTok{runif}\NormalTok{(}\DecValTok{1}\NormalTok{, }\DataTypeTok{min =} \NormalTok{LB[}\DecValTok{5}\NormalTok{], }\DataTypeTok{max =} \NormalTok{UB[}\DecValTok{5}\NormalTok{])       }\CommentTok{# propose from prior}
\NormalTok{P =}\StringTok{ }\KeywordTok{g.predict}\NormalTok{(M.prime, x)                             }\CommentTok{# get predicted values}

\NormalTok{LL.prime =}\StringTok{ }\KeywordTok{sum}\NormalTok{(}\KeywordTok{dnorm}\NormalTok{(y-P, }\DataTypeTok{mean=} \DecValTok{0}\NormalTok{, }\DataTypeTok{sd=}\DecValTok{1}\NormalTok{, }\DataTypeTok{log=}\NormalTok{T))      }\CommentTok{# compute loglikihood}
\NormalTok{M.prime[}\DecValTok{1}\NormalTok{] =}\StringTok{ }\NormalTok{LL.prime                                 }\CommentTok{# save LL}

\NormalTok{r =}\StringTok{ }\KeywordTok{runif}\NormalTok{(}\DecValTok{1}\NormalTok{)                                          }\CommentTok{# random uniform}
\NormalTok{MH =}\StringTok{ }\KeywordTok{exp}\NormalTok{(LL.prime -}\StringTok{ }\NormalTok{LL)                               }\CommentTok{# Metropolis-hasting acceptance probability value}
\NormalTok{if (r <=}\StringTok{ }\NormalTok{MH) \{M =}\StringTok{ }\NormalTok{M.prime\}                            }\CommentTok{# if accepted}

\NormalTok{\}}

\KeywordTok{return}\NormalTok{(M)}
\NormalTok{\}}
\end{Highlighting}
\end{Shaded}

\begin{Shaded}
\begin{Highlighting}[]
\NormalTok{g.death<-function(}\DataTypeTok{M=}\KeywordTok{c}\NormalTok{(-}\OtherTok{Inf}\NormalTok{, }\DecValTok{2}\NormalTok{, }\DecValTok{1}\NormalTok{, }\FloatTok{0.5}\NormalTok{, }\FloatTok{0.0}\NormalTok{), }\DataTypeTok{Qsd=}\KeywordTok{c}\NormalTok{(}\DecValTok{0}\NormalTok{, }\DecValTok{0}\NormalTok{, }\FloatTok{0.1}\NormalTok{, }\FloatTok{0.01}\NormalTok{, }\FloatTok{0.001}\NormalTok{), }\DataTypeTok{LB =} \KeywordTok{c}\NormalTok{(}\DecValTok{0}\NormalTok{, }\DecValTok{1}\NormalTok{, -}\DecValTok{10}\NormalTok{, -}\DecValTok{2}\NormalTok{, -}\DecValTok{1} \NormalTok{), }\DataTypeTok{UB =} \KeywordTok{c}\NormalTok{(}\DecValTok{0}\NormalTok{, }\DecValTok{1}\NormalTok{, }\DecValTok{10}\NormalTok{, }\DecValTok{2}\NormalTok{, }\DecValTok{1} \NormalTok{)  , }\DataTypeTok{data=}\KeywordTok{data.frame}\NormalTok{(}\DecValTok{1}\NormalTok{:}\DecValTok{100}\NormalTok{, }\KeywordTok{rnorm}\NormalTok{(}\DecValTok{100}\NormalTok{)))}
\NormalTok{\{}
\CommentTok{# unpacking hte parameters}
\NormalTok{LL    =}\StringTok{ }\NormalTok{M[}\DecValTok{1}\NormalTok{]}
\NormalTok{j     =}\StringTok{ }\NormalTok{M[}\DecValTok{2}\NormalTok{]}
\NormalTok{x     =}\StringTok{ }\NormalTok{data[,}\DecValTok{1}\NormalTok{]}
\NormalTok{y     =}\StringTok{ }\NormalTok{data[,}\DecValTok{2}\NormalTok{]}

\NormalTok{if (j ==}\StringTok{ }\DecValTok{3}\NormalTok{)}
\NormalTok{\{}
\NormalTok{M.prime =}\StringTok{ }\NormalTok{M                                           }\CommentTok{# make the proposal model}
\NormalTok{M.prime[}\DecValTok{2}\NormalTok{] =}\StringTok{ }\DecValTok{2}
\NormalTok{M.prime[}\DecValTok{5}\NormalTok{] =}\StringTok{ }\DecValTok{0}                                        \CommentTok{# propose from prior}
\NormalTok{P =}\StringTok{ }\KeywordTok{g.predict}\NormalTok{(M.prime, x)                             }\CommentTok{# kill the parameter}

\NormalTok{LL.prime =}\StringTok{ }\KeywordTok{sum}\NormalTok{(}\KeywordTok{dnorm}\NormalTok{(y-P, }\DataTypeTok{mean=} \DecValTok{0}\NormalTok{, }\DataTypeTok{sd=}\DecValTok{1}\NormalTok{, }\DataTypeTok{log=}\NormalTok{T))      }\CommentTok{# compute loglikihood}
\NormalTok{M.prime[}\DecValTok{1}\NormalTok{] =}\StringTok{ }\NormalTok{LL.prime                                 }\CommentTok{# save LL}

\NormalTok{r =}\StringTok{ }\KeywordTok{runif}\NormalTok{(}\DecValTok{1}\NormalTok{)                                          }\CommentTok{# random uniform}
\NormalTok{MH =}\StringTok{ }\KeywordTok{exp}\NormalTok{(LL.prime -}\StringTok{ }\NormalTok{LL)                               }\CommentTok{# Metropolis-hasting acceptance probability value}
\NormalTok{if (r <=}\StringTok{ }\NormalTok{MH) \{M =}\StringTok{ }\NormalTok{M.prime\}                            }\CommentTok{# if accepted}

\NormalTok{\}}

\NormalTok{if (j ==}\StringTok{ }\DecValTok{2}\NormalTok{)}
\NormalTok{\{}
\NormalTok{M.prime =}\StringTok{ }\NormalTok{M                                           }\CommentTok{# make the proposal model}
\NormalTok{M.prime[}\DecValTok{2}\NormalTok{] =}\StringTok{ }\DecValTok{1}
\NormalTok{M.prime[}\DecValTok{4}\NormalTok{] =}\StringTok{ }\DecValTok{0}                                        \CommentTok{# kill the parameter}
\NormalTok{P =}\StringTok{ }\KeywordTok{g.predict}\NormalTok{(M.prime, x)                             }\CommentTok{# get p}

\NormalTok{LL.prime =}\StringTok{ }\KeywordTok{sum}\NormalTok{(}\KeywordTok{dnorm}\NormalTok{(y-P, }\DataTypeTok{mean=} \DecValTok{0}\NormalTok{, }\DataTypeTok{sd=}\DecValTok{1}\NormalTok{, }\DataTypeTok{log=}\NormalTok{T))      }\CommentTok{# compute loglikihood}
\NormalTok{M.prime[}\DecValTok{1}\NormalTok{] =}\StringTok{ }\NormalTok{LL.prime                                 }\CommentTok{# save LL}

\NormalTok{r =}\StringTok{ }\KeywordTok{runif}\NormalTok{(}\DecValTok{1}\NormalTok{)                                          }\CommentTok{# random uniform}
\NormalTok{MH =}\StringTok{ }\KeywordTok{exp}\NormalTok{(LL.prime -}\StringTok{ }\NormalTok{LL)                               }\CommentTok{# Metropolis-hasting acceptance probability value}
\NormalTok{if (r <=}\StringTok{ }\NormalTok{MH) \{M =}\StringTok{ }\NormalTok{M.prime\}                            }\CommentTok{# if accepted}

\NormalTok{\}}

\KeywordTok{return}\NormalTok{(M)}
\NormalTok{\}}
\end{Highlighting}
\end{Shaded}

\begin{Shaded}
\begin{Highlighting}[]
\NormalTok{g.explore<-function(old, d)}
\NormalTok{\{}
\NormalTok{Qsd. =}\StringTok{ }\KeywordTok{c}\NormalTok{(}\DecValTok{0}\NormalTok{, }\DecValTok{0}\NormalTok{, }\FloatTok{0.1}\NormalTok{, }\FloatTok{0.01}\NormalTok{, }\FloatTok{0.001}\NormalTok{)}
\NormalTok{LB.  =}\StringTok{ }\KeywordTok{c}\NormalTok{(}\DecValTok{0}\NormalTok{, }\DecValTok{1}\NormalTok{, -}\DecValTok{10}\NormalTok{, -}\DecValTok{2}\NormalTok{, -}\DecValTok{1} \NormalTok{)}
\NormalTok{UB.  =}\StringTok{ }\KeywordTok{c}\NormalTok{(}\DecValTok{0}\NormalTok{, }\DecValTok{1}\NormalTok{, }\DecValTok{10}\NormalTok{, }\DecValTok{2}\NormalTok{, }\DecValTok{1} \NormalTok{)}

\NormalTok{move.type =}\StringTok{ }\KeywordTok{sample}\NormalTok{(}\DecValTok{1}\NormalTok{:}\DecValTok{3}\NormalTok{, }\DecValTok{1}\NormalTok{) }\CommentTok{# the type of move i.e., perterb, birth, death}

\NormalTok{if (move.type ==}\StringTok{ }\DecValTok{1}\NormalTok{) \{old =}\StringTok{ }\KeywordTok{g.perterb}\NormalTok{(}\DataTypeTok{M=}\NormalTok{old, }\DataTypeTok{Qsd =}\NormalTok{Qsd., }\DataTypeTok{LB =} \NormalTok{LB., }\DataTypeTok{UB=}\NormalTok{UB., }\DataTypeTok{data=} \NormalTok{d)\}}
\NormalTok{if (move.type ==}\StringTok{ }\DecValTok{2}\NormalTok{) \{old =}\StringTok{ }\KeywordTok{g.birth}\NormalTok{(}\DataTypeTok{M=}\NormalTok{old,  }\DataTypeTok{Qsd =}\NormalTok{Qsd., }\DataTypeTok{LB =} \NormalTok{LB., }\DataTypeTok{UB=}\NormalTok{UB., }\DataTypeTok{data =} \NormalTok{d)\}}
\NormalTok{if (move.type ==}\StringTok{ }\DecValTok{3}\NormalTok{) \{old =}\StringTok{ }\KeywordTok{g.death}\NormalTok{(}\DataTypeTok{M=}\NormalTok{old, }\DataTypeTok{Qsd =}\NormalTok{Qsd., }\DataTypeTok{LB =} \NormalTok{LB., }\DataTypeTok{UB=}\NormalTok{UB., }\DataTypeTok{data =} \NormalTok{d )\}}

\KeywordTok{return}\NormalTok{(old)}
\NormalTok{\}}
\end{Highlighting}
\end{Shaded}

\begin{Shaded}
\begin{Highlighting}[]
\NormalTok{g.rjMCMC<-function(}\DataTypeTok{Ndat =} \DecValTok{100}\NormalTok{, }\DataTypeTok{Nsamp =} \DecValTok{25000}\NormalTok{, }\DataTypeTok{BURN =} \DecValTok{1000}\NormalTok{)}
\NormalTok{\{}
\CommentTok{#+ (1:Ndat)*0.75}
\NormalTok{beta0 =}\StringTok{ }\DecValTok{3}
\NormalTok{beta1 =}\StringTok{ }\FloatTok{0.1}
\NormalTok{beta2 =}\StringTok{ }\DecValTok{0}

\NormalTok{data =}\StringTok{ }\KeywordTok{data.frame}\NormalTok{(}\DataTypeTok{x =} \DecValTok{1}\NormalTok{:Ndat, }\DataTypeTok{y =} \NormalTok{beta0  +}\KeywordTok{rnorm}\NormalTok{(Ndat)+}\StringTok{ }\NormalTok{(}\DecValTok{1}\NormalTok{:Ndat)*beta1  ) }\CommentTok{# the simulated data}

\KeywordTok{plot}\NormalTok{(data[,}\DecValTok{1}\NormalTok{], data[,}\DecValTok{2}\NormalTok{], }\DataTypeTok{xlab=}\StringTok{"x"}\NormalTok{, }\DataTypeTok{ylab=}\StringTok{"y"}\NormalTok{, }\DataTypeTok{main =} \StringTok{"Simulated Data"}\NormalTok{)}
\KeywordTok{lines}\NormalTok{(}\DecValTok{1}\NormalTok{:Ndat,beta0  +}\StringTok{  }\NormalTok{(}\DecValTok{1}\NormalTok{:Ndat)*beta1, }\DataTypeTok{col=}\StringTok{"blue"}\NormalTok{, }\DataTypeTok{lwd=}\DecValTok{3} \NormalTok{)}
\KeywordTok{points}\NormalTok{(data[,}\DecValTok{1}\NormalTok{], data[,}\DecValTok{2}\NormalTok{])}

\NormalTok{Mod.old  =}\StringTok{ }\KeywordTok{c}\NormalTok{(-}\OtherTok{Inf}\NormalTok{, }\DecValTok{1}\NormalTok{, }\DecValTok{4}\NormalTok{, }\DecValTok{0}\NormalTok{, }\DecValTok{0}\NormalTok{)}

\NormalTok{for(i in }\DecValTok{1}\NormalTok{:BURN) }\CommentTok{# the burn in}
\NormalTok{\{}
\NormalTok{Mod.old =}\StringTok{ }\KeywordTok{g.explore}\NormalTok{(}\DataTypeTok{old =} \NormalTok{Mod.old, }\DataTypeTok{d =} \NormalTok{data)}
\NormalTok{\}}
\KeywordTok{print}\NormalTok{(Mod.old)}

\NormalTok{REC =}\StringTok{ }\NormalTok{Mod.old}
\NormalTok{for(i in }\DecValTok{1}\NormalTok{:(Nsamp}\DecValTok{-1}\NormalTok{)) }\CommentTok{# the burn in}
\NormalTok{\{}
\NormalTok{Mod.old =}\StringTok{ }\KeywordTok{g.explore}\NormalTok{(}\DataTypeTok{old =} \NormalTok{Mod.old, }\DataTypeTok{d =} \NormalTok{data)}
\NormalTok{REC =}\StringTok{ }\KeywordTok{rbind}\NormalTok{(REC, Mod.old)}
\KeywordTok{rownames}\NormalTok{(REC) =}\StringTok{ }\OtherTok{NULL}
\NormalTok{\}}

\KeywordTok{print}\NormalTok{(}\KeywordTok{table}\NormalTok{(REC[,}\DecValTok{2}\NormalTok{]))}

\NormalTok{x =}\StringTok{ }\DecValTok{16}
\KeywordTok{par}\NormalTok{(}\DataTypeTok{mar =} \KeywordTok{c}\NormalTok{(}\DecValTok{4}\NormalTok{,}\DecValTok{4}\NormalTok{,}\DecValTok{1}\NormalTok{,}\DecValTok{1}\NormalTok{), }\DataTypeTok{oma =} \KeywordTok{c}\NormalTok{(}\DecValTok{1}\NormalTok{,}\DecValTok{1}\NormalTok{,}\DecValTok{1}\NormalTok{,}\DecValTok{1}\NormalTok{))}
\KeywordTok{layout}\NormalTok{(}\DataTypeTok{mat =} \KeywordTok{matrix}\NormalTok{(}\KeywordTok{c}\NormalTok{(}\DecValTok{1}\NormalTok{, }\DecValTok{2}\NormalTok{, }\DecValTok{3}\NormalTok{), }\DataTypeTok{nrow=}\DecValTok{1}\NormalTok{, }\DataTypeTok{ncol=}\DecValTok{3}\NormalTok{, }\DataTypeTok{byrow=}\NormalTok{T) )}

\NormalTok{REC =}\StringTok{ }\KeywordTok{rbind}\NormalTok{(REC, }\KeywordTok{c}\NormalTok{(}\DecValTok{0}\NormalTok{, }\DecValTok{3}\NormalTok{, }\DecValTok{0}\NormalTok{,}\DecValTok{0}\NormalTok{,}\DecValTok{0}\NormalTok{)) }\CommentTok{# just to make the ploting easier}

\NormalTok{H1 =}\StringTok{ }\KeywordTok{hist}\NormalTok{(REC[,}\DecValTok{3}\NormalTok{],}\DataTypeTok{breaks =} \KeywordTok{seq}\NormalTok{(-}\DecValTok{10}\NormalTok{, }\DecValTok{10}\NormalTok{, }\DataTypeTok{length.out =} \DecValTok{1001}\NormalTok{),  }\DataTypeTok{plot=}\NormalTok{F)}
\NormalTok{H2 =}\StringTok{ }\KeywordTok{hist}\NormalTok{(REC[REC[,}\DecValTok{2}\NormalTok{] >=}\StringTok{ }\DecValTok{2} \NormalTok{,}\DecValTok{4}\NormalTok{], }\DataTypeTok{breaks =} \KeywordTok{seq}\NormalTok{(-}\DecValTok{2}\NormalTok{, }\DecValTok{2}\NormalTok{, }\DataTypeTok{length.out =} \DecValTok{1001}\NormalTok{), }\DataTypeTok{plot=}\NormalTok{F)}
\NormalTok{H3 =}\StringTok{ }\KeywordTok{hist}\NormalTok{(REC[REC[,}\DecValTok{2}\NormalTok{] >=}\StringTok{ }\DecValTok{3} \NormalTok{,}\DecValTok{5}\NormalTok{], }\DataTypeTok{breaks =} \KeywordTok{seq}\NormalTok{(-}\DecValTok{1}\NormalTok{, }\DecValTok{1}\NormalTok{, }\DataTypeTok{length.out =} \DecValTok{1001}\NormalTok{), }\DataTypeTok{plot=}\NormalTok{F)}

\KeywordTok{plot}\NormalTok{(H1$mids, H1$den, }\DataTypeTok{type=}\StringTok{"n"}\NormalTok{, }\DataTypeTok{xlab=}\StringTok{"Beta 0"}\NormalTok{, }\DataTypeTok{ylab=} \StringTok{"P(Beta 0)"}\NormalTok{,}\DataTypeTok{xaxs=}\StringTok{"i"}\NormalTok{, }\DataTypeTok{yaxs=}\StringTok{"i"}\NormalTok{)}
\KeywordTok{polygon}\NormalTok{(}\DataTypeTok{x=}\KeywordTok{c}\NormalTok{(H1$mids[}\DecValTok{1}\NormalTok{], H1$mids, H1$mids[}\KeywordTok{length}\NormalTok{(H1$mids)] ), }\DataTypeTok{y=}\KeywordTok{c}\NormalTok{(}\DecValTok{0}\NormalTok{, H1$den, }\DecValTok{0}\NormalTok{), }\DataTypeTok{col=}\StringTok{"grey"}\NormalTok{, }\DataTypeTok{border=}\NormalTok{F  )}
\KeywordTok{abline}\NormalTok{( }\DataTypeTok{v =} \NormalTok{beta0, }\DataTypeTok{col=}\StringTok{"blue"}\NormalTok{, }\DataTypeTok{lwd=}\DecValTok{2}\NormalTok{, }\DataTypeTok{lty=}\DecValTok{2}  \NormalTok{)}

\KeywordTok{plot}\NormalTok{(H2$mids, H2$den, }\DataTypeTok{type=}\StringTok{"n"}\NormalTok{, }\DataTypeTok{xlab=}\StringTok{"Beta 1"}\NormalTok{, }\DataTypeTok{ylab=} \StringTok{"P(Beta 1)"}\NormalTok{,}\DataTypeTok{xaxs=}\StringTok{"i"}\NormalTok{, }\DataTypeTok{yaxs=}\StringTok{"i"}\NormalTok{)}
\KeywordTok{polygon}\NormalTok{(}\DataTypeTok{x=}\KeywordTok{c}\NormalTok{(H2$mids[}\DecValTok{1}\NormalTok{], H2$mids, H2$mids[}\KeywordTok{length}\NormalTok{(H2$mids)] ), }\DataTypeTok{y=}\KeywordTok{c}\NormalTok{(}\DecValTok{0}\NormalTok{, H2$den, }\DecValTok{0}\NormalTok{), }\DataTypeTok{col=}\StringTok{"grey"}\NormalTok{, }\DataTypeTok{border=}\NormalTok{F  )}
\KeywordTok{abline}\NormalTok{( }\DataTypeTok{v =} \NormalTok{beta1, }\DataTypeTok{col=}\StringTok{"blue"}\NormalTok{, }\DataTypeTok{lwd=}\DecValTok{2}\NormalTok{, }\DataTypeTok{lty=}\DecValTok{2}  \NormalTok{)}

\KeywordTok{plot}\NormalTok{(H3$mids, H3$den, }\DataTypeTok{type=}\StringTok{"n"}\NormalTok{, }\DataTypeTok{xlab=}\StringTok{"Beta 2"}\NormalTok{, }\DataTypeTok{ylab=} \StringTok{"P(Beta 2)"}\NormalTok{,}\DataTypeTok{xaxs=}\StringTok{"i"}\NormalTok{, }\DataTypeTok{yaxs=}\StringTok{"i"}\NormalTok{)}
\KeywordTok{polygon}\NormalTok{(}\DataTypeTok{x=}\KeywordTok{c}\NormalTok{(H3$mids[}\DecValTok{1}\NormalTok{], H3$mids, H3$mids[}\KeywordTok{length}\NormalTok{(H3$mids)] ), }\DataTypeTok{y=}\KeywordTok{c}\NormalTok{(}\DecValTok{0}\NormalTok{, H3$den, }\DecValTok{0}\NormalTok{), }\DataTypeTok{col=}\StringTok{"grey"}\NormalTok{, }\DataTypeTok{border=}\NormalTok{F  )}
\KeywordTok{abline}\NormalTok{( }\DataTypeTok{v =} \NormalTok{beta2, }\DataTypeTok{col=}\StringTok{"blue"}\NormalTok{, }\DataTypeTok{lwd=}\DecValTok{2}\NormalTok{, }\DataTypeTok{lty=}\DecValTok{2}  \NormalTok{)}

\NormalTok{\}}
\end{Highlighting}
\end{Shaded}

\end{frame}

\begin{frame}[fragile]{RJMCMC Beispiel Ergebnisse}

\begin{Shaded}
\begin{Highlighting}[]
\KeywordTok{g.rjMCMC}\NormalTok{(}\DataTypeTok{Ndat =} \DecValTok{20}\NormalTok{)}
\end{Highlighting}
\end{Shaded}

\includegraphics{rjmcmc_files/figure-beamer/unnamed-chunk-7-1.pdf}

\begin{verbatim}
## [1] -26.258767   1.000000   4.350712   0.000000   0.000000
## 
##     1     2 
## 24352   648
\end{verbatim}

\includegraphics{rjmcmc_files/figure-beamer/unnamed-chunk-7-2.pdf}

\end{frame}

\end{document}
